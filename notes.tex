\documentclass[12pt]{article}
%\usepackage[left=1in, top=1in, right=1in, bottom=1in]{geometry}
\usepackage[margin=1.5in]{geometry}
\setlength{\marginparwidth}{0.5in}
\setlength{\parindent}{0em}
\usepackage{graphicx}
\usepackage{framed}
\usepackage{tcolorbox}
%\usepackage{tasks}
\usepackage{lipsum}
%\usepackage{enumerate}
\usepackage{enumitem}
\usepackage{amsmath}
\usepackage{amssymb}
%\usepackage{mdwlist}
\usepackage{xcolor}

\usepackage{fancyhdr}
\pagestyle{fancy}
\fancyhf{}  % Clear all headers and footers (including default page number).
\renewcommand{\headrulewidth}{0pt}
\rfoot{\thepage}

\definecolor{mygray}{rgb}{0.43, 0.5, 0.5}
\usepackage{mathtools}
\usepackage{ragged2e}
\newlength\ubwidth
\newcommand\parunderbrace[2]{%
    \settowidth\ubwidth{$\quad#1\quad$}
    \begingroup\color{mygray}\underbrace{\color{black}#1}_{%
    \parbox{\ubwidth}{\scriptsize\centering#2}}\endgroup
}

\usepackage[symbol]{footmisc}
\usepackage{perpage}
\MakePerPage{footnote}
%\renewcommand\footnoterule{\rule{\textwidth}{0.4pt}}
\renewcommand{\footnoterule}{
  \kern -3pt
  \hrule width \textwidth height 0.4pt
  \kern 2pt
}

\usepackage{marginnote}
%\renewcommand*{\raggedrightmarginnote}{\centering}
\renewcommand*{\raggedleftmarginnote}{\centering}
\newcommand{\mar}[1]{\hspace{0pt}\marginpar{-\textcolor{black}{#1}-}}

\definecolor{notmygreen}{rgb}{0.0, 0.42, 0.24}
\definecolor{mygreen}{rgb}{0.0, 0.26, 0.15}
\newcommand{\mynotes}[1]{\textcolor{mygreen}{#1}}

\definecolor{bred}{rgb}{0.8, 0.0, 0.0}

\usepackage{titlesec}
%\titleformat{<command>}
%   [<shape>]{<format>}{<label>}{<sep>}{<before-code>}{<after-code>}
\titleformat{\section}%
    [hang]
  {\filcenter\fontsize{16}{18}\selectfont\bfseries} %\filcenter\bfseries\LARGE
  {\hspace{-0.25in}\arabic{section}.\;}        % label%    {\thesection} %{<label>}
  {1em}     % sep
  {}        % before code
\titleformat{\subsection}%
  {\filcenter\fontsize{14}{16}\selectfont\bfseries} %\filcenter\bfseries\LARGE
  {\arabic{section}.\arabic{subsection}\;}
  {1em}     % sep
  {}        % before code
\titleformat{\subsubsection}%
  {\filcenter\fontsize{13}{15}\selectfont\bfseries\itshape} %\filcenter\bfseries\LARGE
  {\arabic{section}.\arabic{subsection}.\arabic{subsubsection}\;}        % label%    {\thesection} %{<label>}
  {1em}     % sep
  {}        % before code

%\titlespacing*{\section}{-0.5in}{0ex}{0ex}
%\titlespacing*{\subsection}{0pt}{0.5ex}{-10ex}

\titlespacing*{\paragraph}{0ex}{0ex}{2em}
%\titlespacing*{\paragraph}{0pt}{1ex}{-2ex}

% Section references
%\renewcommand{\thesection}{}
%\renewcommand{\thesubsection}{\arabic{subsection}}
%\renewcommand{\thesubsubsection}{\arabic{subsubsection}}

%\setcounter{secnumdepth}{1}

\setitemize{itemsep=-1ex, topsep=0ex,}
\setenumerate{itemsep=-1ex, topsep=0ex,}
\setdescription{itemsep=0ex, align=right,}
\renewcommand{\labelitemi}{$\vcenter{\hbox{\footnotesize$\bullet$}}$}
\renewcommand{\labelitemii}{$\vcenter{\hbox{\footnotesize$\circ$}}$}
%\renewcommand{\labelitemi}{{\tiny$\bullet$}}
\definecolor{cadet}{rgb}{0.33, 0.41, 0.47}
\renewcommand{\descriptionlabel}[1]{%
    \ttfamily\textcolor{cadet}{#1}}

\usepackage{fancyvrb}  % framebox around verbatim text
\makeatletter
\renewcommand\verbatim@font{\normalfont\small\ttfamily}
\makeatother

\usepackage{listings}
\lstset{% general command to set parameter(s)
    basicstyle=\small, % print whole listing small
    keywordstyle=\color{black}\bfseries\underbar,% underlined bold black keywords % nothing happens
    identifierstyle=,
    commentstyle=\color{white},
    stringstyle=\ttfamily,
    showstringspaces=false % no special string spaces
    }

\usepackage{setspace} % spacing between toc items
%\usepackage[toc]{multitoc}
%\renewcommand*{\multicolumntoc}{2}
%\setlength{\columnseprule}{0.5pt}

\usepackage{hyperref}
\definecolor{darkpowderblue}{rgb}{0.0, 0.2, 0.6}
\hypersetup{colorlinks=true,
    urlcolor=darkpowderblue,
    linkcolor=black % This may be what links the contents in the first place!
}
\urlstyle{same}

\begin{document}
\setlength{\parskip}{0pt}
\tableofcontents
\setlength{\parskip}{10pt}
\hypersetup{colorlinks=true, urlcolor=darkpowderblue, linkcolor=darkpowderblue,}
\reversemarginpar

\newpage
\section{Introduction}

\newpage
\section{Radiation from moving charges}
\mar{37}

\subsection{Thomson scattering}
\mar{52}\underline{Process}: response of a free charge (electron) to an
incident EM wave and subsequent scattering of light by the wave.

\underline{Assumption}: if the charge oscillates at non-relativistic velocities,
$v \ll c$, then the magnetic force is negligible
($ \vec{F_{L}} = q (\vec{E} + \cfrac{\vec{v}}{c} \times \vec{B} )  $
and $ |\vec{E}| = |\vec{B}|$ for EM wave).

Consider a linearly polarized incoming wave:

\mar{53}\textcolor{gray}{Does not appear to exist\ldots}

\mar{54}The incident flux of radiation is:
\[
    \langle S \rangle
    = \frac{c}{8\pi} E_{0}^{2}
    \]

\begin{framed}
    \textbf{Thomson cross section for scattering:}
    \[
    \sigma_{TH} = \frac{8\pi}{3} r_{0}^{2}
        \]
\end{framed}

Note:
\begin{enumerate}
    \item $\sigma$ also follows from $P = \langle S \rangle \sigma$
    \item $\sigma$ is independent of frequency \mynotes{(any type:
        radio, optical, etc.)}; however, this is only true in the classical
        case, where $v \ll c$ (see later discussion of Compton scattering).
    \item $\sigma \propto m^{-2} $, so for a proton, $\sigma$ is about $10^{6}$
        times smaller than for an electron, hence in a plasma, electrons
        dominate.
    \item Thomson scattered light introduces a degree of polarization,
        which will be a maximum at $90^{\circ}$ to the incoming beam
        (if the incoming beam is \emph{un}polarized).
        Degree of polarization:
        \[
            \prod = \frac{1-\cos^{2}\theta}{1+\cos^{2}\theta}
            \]
        which is equal to 100\% at $90^{\circ}$.
\end{enumerate}

\subsubsection{Optical depth and Thomson cross section}
\mar{55}When can we still see an object centered on an ionized nebula, which
scatters the radiation only through electron scattering?

\subsubsection{Additional comments on Thomson scattering}
\mar{56}Thomson scattering is a special case of the ``collision'' between
a photon and an electron. It applies when the energy of the photon is
small compared to the rest energy of the electron. When this is not the
case, the situation is analyzed from the perspective of Compton scattering.
A Compton scattering analysis also takes into effect quantum corrections,
e.g.\ kinematic effects occur because the photon not only possesses energy
but also momenutum.\footnote{R/L chapter 7 discusses this in great detail.}

\subsection{Eddington limit}
\mar{57}Following R/L problem 1.4 parts a \& c:
\begin{itemize}[label={}]
    \item a.) ``Show that an optically thin cloud\ldots''
    \item c.) ``A minimum value for\ldots''
\end{itemize}

\mar{58}

\mar{59}We essentially have\ldots

\underline{Applications}:
\begin{itemize}
    \item There is a limit to how massive a star can be:
        \[
            L_{Edd} \sim 1.25 \times 10^{38} \quad
            \left[
                \mathrm{erg}\;\mathrm{s}^{-1} \left( \frac{M}{M_{\odot}} \right)
                \right]
            \]
        If $L$ for MS stars scales as $ L \propto M^{x} $, this places a limit
        on the highest mass star we can have ($\sim$ 100 M$_{\odot}$).
    \item Accretion on black holes and accretion disks: note that $L_{Edd}$
        scales linearly with $M$, and is independent of $r$
\end{itemize}

\subsection{Plasma effects}
\mar{60}General issue: what happens to EM waves when they travel through
a plasma?

\mar{61}

\mar{62}

\underline{Application}: \textbf{pulsar dispersion measurements}

A pulsar is a fast spinning neutron star emitting two beams of radiation
(like a lighthouse). If there is an ionized plasma along the line of sight
to the pulsar, radiation in the pulse of lower frequency will reach us later
than high frequency radiation because it is slowed down more by the
plasma.\footnote{
    see expression for $v_{ph}$}
(in a sense, it is scattered more, therefore travels a greater distance).

\subsection{Faraday rotation}
\mar{69}Basic idea: what happens to radiation (EM wave) travelling through
a cold plasma if the plasma has a uniform magnetic field in it?
In particular, what happens to a linearly polarized wave?
As we will see, the direction of polarization changes!

\newpage
\section{Fourier transforms and convolutions}
\mar{F1}References: Bracewell - \textit{The FT and its applications}
\subsection{Fourier transforms}
Basic idea goes back to Fourier series, which approximate a general function
as a sum of sines and cosines of different periods.

A simple example is a vibrating string, which can be represented as:
\[
    g(x) = a_{0} + \sum_{1}^{\infty} \left[
        a_{n}\cos{2 \pi nfx} + b_{n}\sin{2 \pi nfx} \right]
    \]
which describes a periodic function $g(x)$ with period $T$ and frequency $f$,
where
\[
    a_{0} = \frac{1}{T} \int_{-\frac{1}{2}T}^{+\frac{1}{2}T}{
        g(x) \mathrm{d}x}
    \]
\[
    a_{n} = \frac{2}{T} \int_{-\frac{1}{2}T}^{+\frac{1}{2}T}{
        g(x) \cos{2 \pi nfx} \mathrm{d}x}
    \]
etc. The general definition of a fourier transform (FT) is
\[
    F(s) = \int_{-\infty}^{+\infty}{
        f(x) e^{-i2{\pi}xs} \mathrm{d}x }
    \]
\[
    f(x) = \int_{-\infty}^{+\infty}{
        F(s) e^{+i2{\pi}xs} \mathrm{d}s }
    \]
The two transforms are not exactly equivalent because of the $\pm$ sign.
If this were not present, we would not necessarily get back the original
$f(x)$ in the second transform.\footnote{See Bracewell.}

Note here the link with the sine and cosine series, since
\[
    e^{i\theta} = \cos\theta + i\sin\theta
    \]

\mar{F2}Other conventions for where to put the $2\pi$ exist:
\[
    F(s) = \int_{-\infty}^{\infty}{
        f(x)e^{-ixs} \mathrm{d}x}
    \quad\longleftrightarrow\quad
    f(x) = \frac{1}{2\pi} \int_{-\infty}^{\infty}{
        F(s)e^{ixs} \mathrm{d}s}
    \]
\[
    F(s) = \frac{1}{\sqrt{2\pi}} \int_{-\infty}^{\infty}{
        f(x)e^{-ixs} \mathrm{d}x}
    \quad\longleftrightarrow\quad
    f(x) = \frac{1}{\sqrt{2\pi}} \int_{-\infty}^{\infty}{
        F(s)e^{ixs} \mathrm{d}s}
    \]
If $f(x)$ and $F(s)$ are transform pair in our system, then $f(x)$,
$F \left( \cfrac{s}{2\pi} \right)$ are pair in second system and
$f \left( \cfrac{x}{\sqrt{2\pi}} \right)$,
$F \left( \cfrac{s}{\sqrt{2\pi}} \right)$ in third system.

\subsubsection{Symmetry: oddness and evenness}
Symmetrical function: if $E(x) = E(-x)$, then $E(x)$ is symmetrical,
or even. If $O(x) = -O(-x)$, then $O(x)$ is an odd function.
\mar{F3}Verify: sum of an even function and an odd function is typically
neither even nor odd.


\subsection{Complex conjugates}
\mar{F6}The FT of\ldots

\subsection{Cosine and sine transforms}

\mar{F7}The cosine and

\subsection{Convolution}

\mar{F10}

\subsection{Auto-correlation function}

\subsection{Impulse Symbol}
\mar{F15}Another somewhat bizarre but useful function is the \textit{impulse
symbol}. By that we mean an intense, unit-area pulse of length so short that it
cannot be resolved with a given piece of measuring equipment. Essentially, a
pulse like this will show up only in integrals (hence its unit-area) and it is
equivalent to a $\delta$-function, which is the mathematical analog.
$\delta$-functions permeate (theoretical) physics: point masses, point charges,
point sources, line sources, surface charges, etc. These things don't exist,
but your measuring equipment doesn't know this, anything smaller than its
resolution (in time, space, or frequency), acts as an impulse function, in a
mathematical sense.

\mar{F24}Properties related to the convolution theorem:
\begin{enumerate}
    \item The area under a convolution is equal to the product of the
        areas under the factors:
        \[
            \int_{-\infty}^{\infty} \left( f*g \right) \mathrm{d}x
            \quad = \quad
            \left[ \int_{-\infty}^{\infty} f(x)\mathrm{d}x \right]
            \left[ \int_{-\infty}^{\infty} g(x)\mathrm{d}x \right]
            \]
    \item The abscissas of the centers of gravity add:
        \[
            \]
    \item The second moments add if\ldots
        This has the important consequence that under convolution,
        typically the variances add.
\end{enumerate}

\newpage
\section{Signal processing and receivers: theory}

\newpage
\section{Practical receiver systems}

\newpage
\section{Practical aspects of filled aperture antennas}

\newpage
\section{Single dish observational methods}
\subsection{Atmosphere}
\subsection{Calibration procedures}
\subsection{Continuum observing strategies}
\subsection{Spectral line observations: additional requirements}

\newpage
\section{Interferometers and aperture synthesis}

\newpage
\section{Radiation processes}
\mar{145}
\subsection{Bremsstrahlung radiation}
= free-free radiation\footnote{following: Rybicki \& Lightman, chapter 5.}

Process: ionized plasma, encounters between electrons and ions causes deflections
of electrons $\rightarrow$ electrons are accelerated and will radiate.
Under most conditions, velocity distributions of ions and electrons is
\textit{Maxwellian} - thermal equilibrium (TE) distribution, so this is referred
to as \textit{thermal Bremsstrahlung}. The word ``Bremsstrahlung'' means
``braking radiation'', which refers to the slowing down (deceleration) of the
electrons when approaching from a given direction to the protons/ions. The approach
will be classical, rather than quantum mechanical; this is valid as long as
$ kT \sim mv_{e}^{2} \gg h\nu$.

Note: collisions between the same particles produce \emph{no net} dipole radiation
because dipole moment = $\sum{e_{i}\vec{r}_{i}} \propto \sum{m_{i}\vec{r}_{i}}$.
The latter is the center of mass, which is either at rest or in constant motion,
but experiences no \emph{net} acceleration. In other words, each particle moves
exactly opposite to its partner so that their individual contributions to the
time-varying electric fields at large distances cancels\footnote{Shu}.

\subsubsection{Emission for single-speed electrons}
\mar{146}
Here we consider electron-ion collisions (or rather, \emph{encounters}).
In these encounters, the electrons radiate primarily since
$ |\vec{a}| \propto \cfrac{1}{m} $ for equal charges.

\underline{Approximations}:
\begin{enumerate}
    \item ion mass $\gg$ electron mass $\rightarrow$ electron moves in \emph{fixed}
        Coulomb field of ion.
    \item electron speed is large; in a single encounter, the electron motion barely
        deviates from its original direction $\rightarrow$ small angle scattering regime
        (not essential approximation).
\end{enumerate}
Take FT, note that\footnote{RL 3.3}
\begin{align*}
    FT (\ddot{\vec{d}})
    &= -\omega^{2} \hat{\vec{d}} (\omega)\\
    &= -\frac{e}{2\pi} \int_{-\infty}^{\infty}{
        \dot{\vec{v}} e^{i\omega{t}} \mathrm{d}t }
\end{align*}
Derive expressions for $\hat{\vec{d}}(\omega)$ in limits for large and small $\omega$:
collision time $\tau \equiv \cfrac{b}{v}$.

\mar{147}We had derived before the spectrum of dipole radiation:
\[
    \frac{\mathrm{d}W}{\mathrm{d}\omega}
    = \frac{8\pi\omega^{4}}{3c^{3}} | \hat{d}(\omega) |^{2}
    \]
So now we find
\[
    \]
No dependence on $\omega$! (Impulse has flat spectrum).
\underline{Clearly the task is to estimate $\Delta\vec{v}$}.
We use our approximation that the path of the electron is almost linear.
$\Delta\vec{v}$ is mainly in $\perp$ direction.
\[
    \]
(Use Coulomb force: $ F = m_{e}\cfrac{\Delta{v}}{\Delta{t}} = \cfrac{Ze^{2}}{R^{2}}$.)

\subsubsection{Determining the total spectrum}

\mar{148}

\mar{149}

We deal with the $b_{min}, b_{max}$ issue, and the potential importance of
quantum effects, by introducing the \textbf{Gaunt factor}, $g_{ff}(v,\omega)$,
which includes these effects.
\subsubsection{Velocity distribution}

\subsection{Synchrotron radiation}
\mar{167}

\end{document}
