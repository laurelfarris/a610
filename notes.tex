\documentclass[12pt]{article}
%\usepackage[left=1in, top=1in, right=1in, bottom=1in]{geometry}
\usepackage[margin=1.5in]{geometry}
\setlength{\marginparwidth}{0.5in}
\setlength{\parindent}{0em}
\setlength{\parskip}{2ex}
\usepackage{graphicx}
\usepackage{tcolorbox}
%\usepackage{tasks}
\usepackage{lipsum}
%\usepackage{enumerate}
\usepackage{enumitem}
\usepackage{amsmath}
\usepackage{amssymb}
%\usepackage{mdwlist}
\usepackage{xcolor}

\usepackage{fancyhdr}
\pagestyle{fancy}
\fancyhf{}  % Clear all headers and footers (including default page number).
\renewcommand{\headrulewidth}{0pt}
\rfoot{\thepage}

\definecolor{mygray}{rgb}{0.43, 0.5, 0.5}
\usepackage{mathtools}
\usepackage{ragged2e}
\newlength\ubwidth
\newcommand\parunderbrace[2]{%
    \settowidth\ubwidth{$\quad#1\quad$}
    \begingroup\color{mygray}\underbrace{\color{black}#1}_{%
    \parbox{\ubwidth}{\scriptsize\centering#2}}\endgroup
}

\usepackage[symbol]{footmisc}
\usepackage{perpage}
\MakePerPage{footnote}
%\renewcommand\footnoterule{\rule{\textwidth}{0.4pt}}
\renewcommand{\footnoterule}{
  \kern -3pt
  \hrule width \textwidth height 0.4pt
  \kern 2pt
}

\usepackage{marginnote}
%\renewcommand*{\raggedrightmarginnote}{\centering}
\renewcommand*{\raggedleftmarginnote}{\centering}
\newcommand{\mar}[1]{\hspace{0pt}\marginpar{-\textcolor{black}{#1}-}}

\definecolor{notmygreen}{rgb}{0.0, 0.42, 0.24}
\definecolor{mygreen}{rgb}{0.0, 0.26, 0.15}
\newcommand{\mynotes}[1]{\textcolor{mygreen}{#1}}

\definecolor{bred}{rgb}{0.8, 0.0, 0.0}

\usepackage{titlesec}
%\titleformat{<command>}
%   [<shape>]{<format>}{<label>}{<sep>}{<before-code>}{<after-code>}
\titleformat{\section}%
    [hang]
  {\filcenter\fontsize{16}{18}\selectfont\bfseries} %\filcenter\bfseries\LARGE
  {\hspace{-0.25in}\arabic{section}.\;}        % label%    {\thesection} %{<label>}
  {1em}     % sep
  {}        % before code
\titleformat{\subsection}%
  {\filcenter\fontsize{14}{16}\selectfont\bfseries} %\filcenter\bfseries\LARGE
  {\arabic{section}.\arabic{subsection}\;}
  {1em}     % sep
  {}        % before code
\titleformat{\subsubsection}%
  {\filcenter\fontsize{13}{15}\selectfont\bfseries\itshape} %\filcenter\bfseries\LARGE
  {\arabic{section}.\arabic{subsection}.\arabic{subsubsection}\;}        % label%    {\thesection} %{<label>}
  {1em}     % sep
  {}        % before code

%\titlespacing*{\section}{-0.5in}{0ex}{0ex}
%\titlespacing*{\subsection}{0pt}{0.5ex}{-10ex}

\titlespacing*{\paragraph}{0ex}{0ex}{2em}
%\titlespacing*{\paragraph}{0pt}{1ex}{-2ex}

% Section references
%\renewcommand{\thesection}{}
%\renewcommand{\thesubsection}{\arabic{subsection}}
%\renewcommand{\thesubsubsection}{\arabic{subsubsection}}

%\setcounter{secnumdepth}{1}

\setitemize{itemsep=-1ex, topsep=0ex,}
\setenumerate{itemsep=-1ex, topsep=0ex,}
\setdescription{itemsep=0ex, align=right,}
\renewcommand{\labelitemi}{$\vcenter{\hbox{\footnotesize$\bullet$}}$}
\renewcommand{\labelitemii}{$\vcenter{\hbox{\footnotesize$\circ$}}$}
%\renewcommand{\labelitemi}{{\tiny$\bullet$}}
\definecolor{cadet}{rgb}{0.33, 0.41, 0.47}
\renewcommand{\descriptionlabel}[1]{%
    \ttfamily\textcolor{cadet}{#1}
}



% Verbatim
\usepackage{fancyvrb}  % framebox around verbatim text
\makeatletter
\renewcommand\verbatim@font{\normalfont\small\ttfamily}
\makeatother

\usepackage{listings}
\lstset{% general command to set parameter(s)
    basicstyle=\small, % print whole listing small
    keywordstyle=\color{black}\bfseries\underbar,% underlined bold black keywords % nothing happens
    identifierstyle=,
    commentstyle=\color{white},
    stringstyle=\ttfamily,
    showstringspaces=false % no special string spaces
    }

\usepackage{setspace} % spacing between toc items
\usepackage[toc]{multitoc}
\renewcommand*{\multicolumntoc}{2}
%\setlength{\columnseprule}{0.5pt}

\usepackage{hyperref}
\definecolor{darkpowderblue}{rgb}{0.0, 0.2, 0.6}
\hypersetup{colorlinks=true,
    urlcolor=darkpowderblue,
    linkcolor=black % This may be what links the contents in the first place!
}
\urlstyle{same}

\begin{document}
\tableofcontents

\reversemarginpar

\newpage
\section{Introduction}

\newpage
\section{Radiation from moving charges}
\mar{37}

\newpage
\section{Fourier transforms and convolutions}

\newpage
\section{Signal processing and receivers: theory}

\newpage
\section{Practical receiver systems}

\newpage
\section{Introduction}

\newpage
\section{Introduction}

\newpage
\section{Practical aspects of filled aperture antennas}

\newpage
\section{Single dish observational methods}
\subsection{Atmosphere}
\subsection{Calibration procedures}
\subsection{Continuum observing strategies}
\subsection{Spectral line observations: additional requirements}

\newpage
\section{Interferometers and aperture synthesis}

\newpage
\section{Radiation processes}
\mar{145}
\subsection{Bremsstrahlung radiation}
= free-free radiation\footnote{following: Rybicki \& Lightman, chapter 5.}

Process: ionized plasma, encounters between electrons and ions causes deflections
of electrons $\rightarrow$ electrons are accelerated and will radiate.
Under most conditions, velocity distributions of ions and electrons is
\textit{Maxwellian} - thermal equilibrium (TE) distribution, so this is referred
to as \textit{thermal Bremsstrahlung}. The word ``Bremsstrahlung'' means
``braking radiation'', which refers to the slowing down (deceleration) of the
electrons when approaching from a given direction to the protons/ions. The approach
will be classical, rather than quantum mechanical; this is valid as long as
$ kT \sim mv_{e}^{2} \gg h\nu$.

Note: collisions between the same particles produce \emph{no net} dipole radiation
because dipole moment = $\sum{e_{i}\vec{r}_{i}} \propto \sum{m_{i}\vec{r}_{i}}$.
The latter is the center of mass, which is either at rest or in constant motion,
but experiences no \emph{net} acceleration. In other words, each particle moves
exactly opposite to its partner so that their individual contributions to the
time-varying electric fields at large distances cancels\footnote{Shu}.

\subsubsection{Emission for single-speed electrons}
\mar{146}
Here we consider electron-ion collisions (or rather, \emph{encounters}).
In these encounters, the electrons radiate primarily since
$ |\vec{a}| \propto \cfrac{1}{m} $ for equal charges.

\underline{Approximations}:
\begin{enumerate}
    \item ion mass $\gg$ electron mass $\rightarrow$ electron moves in \emph{fixed}
        Coulomb field of ion.
    \item electron speed is large; in a single encounter, the electron motion barely
        deviates from its original direction $\rightarrow$ small angle scattering regime
        (not essential approximation).
\end{enumerate}
Take FT, note that\footnote{RL 3.3}
\begin{align*}
    FT (\ddot{\vec{d}})
    &= -\omega^{2} \hat{\vec{d}} (\omega)\\
    &= -\frac{e}{2\pi} \int_{-\infty}^{\infty}{
        \dot{\vec{v}} e^{i\omega{t}} \mathrm{d}t }
\end{align*}
Derive expressions for $\hat{\vec{d}}(\omega)$ in limits for large and small $\omega$:
collision time $\tau \equiv \cfrac{b}{v}$.

\mar{147}We had derived before the spectrum of dipole radiation:
\[
    \frac{\mathrm{d}W}{\mathrm{d}\omega}
    = \frac{8\pi\omega^{4}}{3c^{3}} | \hat{d}(\omega) |^{2}
    \]
So now we find
\[
    \]
No dependence on $\omega$! (Impulse has flat spectrum).
\underline{Clearly the task is to estimate $\Delta\vec{v}$}.
We use our approximation that the path of the electron is almost linear.
$\Delta\vec{v}$ is mainly in $\perp$ direction.
\[
    \]
(Use Coulomb force: $ F = m_{e}\cfrac{\Delta{v}}{\Delta{t}} = \cfrac{Ze^{2}}{R^{2}$.)

\subsubsection{Determining the total spectrum}

\mar{148}

\mar{149}

We deal with the $b_{min}, b_{max}$ issue, and the potential importance of quantum effects,
by introducing the \textbf{Gaunt factor}, $g_{ff}(v,\omega)$, which includes these effects.
\subsubsection{Velocity distribution}

\subsection{Synchrotron radiation}
\mar{167}

\end{document}
